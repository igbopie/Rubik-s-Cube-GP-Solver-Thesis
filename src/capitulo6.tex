\chapter{Participando en GECCO 2009}\label{ch:gecco2009}

The Genetic and Evolutionary Computation Conference (GECCO) es una de las 11
conferencias internacionales más importantes de inteligencia
artificial. En la conferencia se presentan los últimos resultados en el campo
en crecimiento de la computación evolutiva. Además se suelen hacer
competiciones donde la gente expone sus soluciones a los problemas dados.

Parabon es una empresa que se dedica a la computación distribuida a través de
su plataforma Frontier. Con animo de promocionar su plataforma, Parabon
patrocina una competición del GECCO 2009. Se trata de evolucionar un programa
que consiga resolver cualquier cubo de Rubik en el menor número de pasos
posibles.

La competición de Parabon exigía en sus bases que el programa se evolucionase
en su plataforma Frontier. Además Parabon ofrece otra plataforma para
desarrollar programación genética llamado Origin. Origin esta basado en ECJ,
una plataforma de computación evolutiva, adaptando ECJ para funcionar sobre
Frontier.

Al intentar utilizar Origin descubrimos que la plataforma estaba en vias de
desarrollo y resultaba casi imposible trabajar con ella. Es por esto que
migramos nuestro desarrollo a ECJ.

A partir de las experiencias anteriores se diseñó un experimento que tenía como
objetivo la generación de un programa capaz de tomar parte en la competición.

La simulación se llevó a cabo con los parámetros anteriormente determinados como
óptimos y con cubos de un máximo de 10 movimientos de desorden y con una cantidad
total de 2000 cubos diferentes aproximadamente. Este límite se debe a las
limitaciones del hardware en que tuvo lugar la misma.

La ejecución de la simulación se extendió durante 15 días y los resultados de la
misma fueron enviados a la competición. A pesar de que la competición fue
suspendida por falta de participantes válidos, la solución aportada fue
reconocida por los organizadores como la mejor participación. Los resultados de
la misma se resumen en la tabla \ref{tab:bestsolver}. Así mismo el programa
enviado se muestra en el apéndice \ref{sec:mejor-individuo}.


\begin{table}[tb]
\caption{Resultados de funcionamiento del mejor individuo.}
\label{tab:bestsolver}
\centering
\begin{tabular}{lccr}
\toprule
 \textbf{Dificultad} & \textbf{Resueltos} & \textbf{Total}&
  \textbf{Porcentaje}\\
\midrule
1 & 12 & 12 & 100.00\% \\
2&135&144&93.75\%\\
3&589&860&68.49\%\\
4&373&1000&37.30\%\\
5&148&1000&14.80\%\\
6&37&1000&3.70\%\\
7&12&1000&1.20\%\\
8&1&1000&0.10\%\\
9&2&1000&0.20\%\\
10&1&1000&0.10\%\\
\bottomrule
\textbf{Total}&1310&8016&16.34\%\\
\bottomrule
\end{tabular}
\end{table}



