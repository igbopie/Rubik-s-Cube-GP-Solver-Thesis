\chapter{Conclusiones}\label{ch:concl}

En el presente trabajo hemos abordado el problema de la generación automática,
mediante programación genética, de programas capaces de resolver cubos de Rubik.
Con este fin hemos definido un lenguaje apropiado para representar la solución
del problemay formulado una función de fitness adecuada para que nuestro sistema
evolutivo consiga desarrollarse y generar buenas soluciones. También hemos
propuesto un algoritmo de evaluación que obtenga datos suficientes para
determinar el rendimiento de programa que se está evolucionando.

Sobre la base de estas propuestas hemos realizado un estudio de los parámetros
del algoritmo evolutivo a la vez que se realizaron varias optimizaciones del
código usado.

En estos experimentos se logró tener soluciones capaces de resolver algunos cubos
de alto grado de desorden, en particular cubos que necesitaban hasta 10
movimientos para ser resueltos. Debido a las restricciones en cuanto a recursos
computacionales y tiempo dedicados a este trabajo fue imposible continuar
llevando a cabo estos experimentos. Hay que hacer notar que en un principio se
contaba con poder usar la plataforma de computación distribuida creada por
Parabon pero que esto no fue posible y solo se tuvo a disposición recursos
locales de cómputo.

A partir de las experiencias obtenidas en el estudio mencionado se evolucionó un
programa que fue enviado a la \textit{Parabon Sponsored GECCO Competition}.

A pesar de los inconvenientes prácticos que han impedido el completo despliegue
de la propuesta planteada se han obtenido varios resultados de interés. En
particular, se ha logrado una mejor comprensión de las características del
problema y se han identificado posibles líneas de solución.

Este trabajo puede ser mejorado de varios puntos de vista. Resulta de interés
comprobar las consecuencias de introducir elementos de optimización
multiobjectivo para así lograr una mejor representación del fitness.

Otro aspecto interesante sería introducir el concepto de subpoblaciones en el
proyecto. De esta forma, existirían varias poblaciones coexistiendo y
evolucionando paralelamente. En determinados momentos, estas subpoblaciones pueden
intercambiarse un número determinado de individuos. De esta forma, se aportaría
una nueva fuente de diversidad en las poblaciones.

También resulta importante encontrar un contexto adecuado para la ejecución de
los experimentos, ya sea en la plataforma de Parabon u otro ordenador o grupo de
ordenadores de alto desempeño.

Aunque no se haya conseguido crear un resolvedor universal de cubos de Rubik,
hemos asentado una base y unas lineas futuras de investigación con las que
esperamos que algún día el problema sea resuelto venciendo a los resolvedores
actuales creados mediante métodos de busqueda de todo el espacio de soluciones.
