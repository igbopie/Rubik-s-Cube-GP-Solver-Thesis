\chapter{Objetivos del PFC}\label{ch:objetivos}

El objetivo del proyecto de fin de carrera consiste en evolucionar un programa a
través de la programación genética capaz de resolver el cubo de Rubik de forma
óptima. Como este objetivo es realmente amplio, se dividirá en varios
subobjetivos:

\section{Encontrar un lenguaje apropiado para representar la solución del
problema}

El lenguaje que utilice nuestro programa necesitando pocos nodos, será capaz  de
expresar un solución. Cuanto menos nodos tenga que generar la programación, la
soluciones serán menos complejas, por lo tanto, la convergencia del algoritmo
será más alta, llegando a una solución más rápidamente.

\section{Encontrar un fitness adecuado para que nuestro sistema evolutivo
consiga desarrollarse y generar soluciones del cubo de Rubik.}

La elección del fitness es fundamental para la correcta evolución del sistema.
Tiene que mantener “sana” la población y a la vez indicar claramente qué
individuos son mejores que otros.

\section{Encontrar un algoritmo de evaluación que obtenga datos
suficientes para determinar el rendimiento del programa.}

El algoritmo de evaluación se encargará de probar el programa con una muestra
representativa del problema, lidiando con factores de eficiencia y rendimiento,
para conseguir en el menor tiempo posible sacar una muestra fiable de las
capacidades del programa evaluado.

\section{Encontrar los parámetros adecuados de nuestro framework para optimizar el
desarrollo del resolvedor.}

Nuestro framework de trabajo contiene infinidad de parámetros a configurar, los
cuales requieren ser correctamente establecidos, ya que una mala configuración
puede ser fatal y generar programas inoperativos. La configuración es crucial
para el éxito de nuestro sistema evolutivo.

\section{Optimizar la velocidad de evaluación.}

Para poder obtener soluciones en un tiempo razonable, el sistema tendrá que
ahorrar cualquier tiempo de CPU extra. Para ello se necesitará crear multitud de
cachees y mecanismos que eviten la repetición de procesos que dan lugar al mismo
resultado ya calculado anteriormente.

\section{Resolver cualquier cubo de Rubik.}

Dentro del objetivo del proyecto, antes de conseguir encontrar una solución
óptima al problema, necesitaremos encontrar una solución cualquiera del problema.
Y, a partir de ahí, reducir el número de pasos a realizar para resolver el cubo.

\section{Buscar las soluciones más compactas.}

Una vez que tenemos la solución al problema, los programas tendrán que ir bajando
la complejidad de sus soluciones hasta llegar a un cierto limite. Obviamente se
desconoce cual es éste, sin embargo, nuestro límite será el tiempo que
dispongamos para encontrar una solución menor. Además llegará un punto en el que
apenas mejore. En ese punto podremos dar por finalizada la ejecución.
